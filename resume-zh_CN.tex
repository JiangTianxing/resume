% !TEX TS-program = xelatex
% !TEX encoding = UTF-8 Unicode
% !Mode:: "TeX:UTF-8"

\documentclass{resume}
\usepackage{zh_CN-Adobefonts_external}
\usepackage{linespacing_fix}
\usepackage{cite}


\begin{document}
  \pagenumbering{gobble}

  \name{蒋天星}

  \basicInfo{
  \github[JiangTianxing]{https://github.com/jiangtianxing} \textperiodcentered\
  \email{jx3536@163.com} \textperiodcentered\
  \phone{(+86) 177-9106-3378}}

\section{\faGraduationCap\  教育背景}
  \datedsubsection{\textbf{电子科技大学}}{2020年9月 -- 至今}
  \textit{硕士}\ 计算机学院, 计算机技术
  \datedsubsection{\textbf{重庆邮电大学}}{2015年9月 -- 2020年6月}
  \textit{学士}\ 计算机学院, 智能科学与技术

\section{\faCogs\ 个人经历}

  \datedsubsection{\textbf{红岩网校工作站}, 重庆邮电大学}{2017年6月 -- 2018年6月}
  \role{\textbf{工作站站长}}{}
  \begin{onehalfspacing}
    工作站成立于 2000 年,是拥有 70 余名成员的校园\textbf{互联网团队}。拥有\textbf{产品-视觉-前后端开发-运维} 完整开发流程,努力改善校内互联网环境,为师生提供便捷线上服务。
    工作期间,负责管理各部门日常工作,开展150余次\textbf{技术培训},领导20余项\textbf{应用开发}与\textbf{线上活动策划}。新媒体产品多次受到重庆各大媒体报道,并被重庆团市委积极转载,团队
    获得``\textbf{全国优秀共青团支部}"等荣誉。
  \end{onehalfspacing}

  \datedsubsection{\textbf{网络空间中心}, 鹏城实验室}{2020年12月 -- 2021年3月}
  \role{\textbf{学术实习生}}{}
  \begin{onehalfspacing}
    国家级重点实验室,参与\textbf{鹏城靶场}项目,负责基于 MDATA 模型的网络安全知识图谱研究工作。
  \end{onehalfspacing}

\section{\faHeartO\ 获得荣誉}
  \datedline{\textit{校级},重庆邮电大学\textbf{第七届``五四之星"}获得者(校园最高奖项),\textbf{网络建设之星}}{2018年5月}
  \datedline{\textit{校级},重庆邮电大学\textbf{``优秀共青团干部"}}{2018年5月}
  \datedline{\textit{全国}, \textbf{``创青春"双创杯全国大学生创新创业大赛}, 铜奖}{2018年6月}

  % \datedsubsection{\textbf{仁荷大学}, 韩国仁川}{2018年7月 -- 2018年8月}
  % \role{交换生}{}
  % \begin{onehalfspacing}
  %   重邮选派学生暑假赴韩国仁荷大学交换学习项目。探究韩国近现代历史、文化与国际关系等课题。
  % \end{onehalfspacing}

  % \datedsubsection{\textbf{鹏城实验室}, 网络安全中心}{2021年1月 -- 2021年3月}
  % \role{实习生}{}
  % \begin{onehalfspacing}
  %   参与“鹏城靶场”项目,负责基于 MDATA 模型的网安知识图谱研究调研工作。
  % \end{onehalfspacing}

  % \datedsubsection{\textbf{重庆芯承索屹科技有限公司}, 重庆}{2019年1月 -- 2019年5月}
  % \role{实习 服务端工程师}{产品部}
  % \begin{onehalfspacing}
  %   重邮创新创业基地团队。尝试自主创业,完成多个商业项目,负责各类产品服务端技术支持。
  % \end{onehalfspacing}

\section{\faUsers\ 项目实践}

  % \datedsubsection{\textbf{ScamHunter}, 蚂蚁金服学术合作项目}{2022年~至今}
  % \role{框架开发}{}
  % \begin{onehalfspacing}
  %   端到端的欺诈型DeFi应用检测框架。基于前端自动化测试框架与静态分析方法,实现了前端\textbf{可执行路径}与上下文语义的提取。通过自动化执行前端操作事件序列触发应用潜在的欺诈行为,并基于钱包和区块链插桩获取相关状态信息,实现基于前端操作事件与区块链交易语义一致性判断的欺诈型DeFi应用检测流程。
  % \end{onehalfspacing}
  \datedsubsection{\textbf{Norbert/LagrangeX}, 字节跳动实习}{2022年}
  \role{Data-AML-Engine, 机器学习平台研发}{}
  \begin{onehalfspacing}
  % 面向字节推广搜等业务的大规模训练与推理需求,提供统一的机器学习中台服务,以提升模型迭代效率;在职期间,围绕 Norbert 训练调度框架,提出\textbf{社区化重构}方案设计,接入\textbf{TensorBoard Analyzer}指标分析检测功能,参与\textbf{Godel云原生迁移上量}支持、\textbf{训练调度稳定性}建设相关工作。

  面向字节推广搜等业务的大规模训练与推理需求,提供统一的机器学习中台服务,以提升模型迭代效率;在职期间,围绕 Norbert 训练调度框架,基于状态机模型重构训练\textbf{任务生命周期管理}模块;接入\textbf{TensorBoard Analyzer}指标分析监测功能;参与\textbf{Godel云原生迁移}、\textbf{训练调度稳定性}建设相关工作。
  \end{onehalfspacing}

  \datedsubsection{\textbf{AntFuzzer}, 蚂蚁金服学术合作}{2021年}
  \role{数字科技线,区块链安全工具开发}{}
  \begin{onehalfspacing}
    以代码覆盖率为导向的EOSIO智能合约灰盒模糊测试工具。工具由AntFuzzer、EOSIO区块链、模糊测试框架AFL等三大组件构成;基于\textbf{消息队列}与\textbf{共享内存}等进程间通信方式,融合了AFL作为\textbf{跨平台}参数生成引擎;基于智能合约生命周期与WASM指令解释执行过程,针对EOSIO区块链的链API函数与WASM虚拟机插桩,获取运行时链上状态,并计算智能合约\textbf{代码覆盖率};在AntFuzzer中,基于\textbf{IOC容器}实现\textbf{漏洞检测方案}与测试环境构建、组件间通信、测试用例生成、智能合约函数调用等\textbf{自动化检测流程}的分离;根据EOSIO智能合约漏洞的攻击原理,实现了八种主流漏洞的检测方案。
  \end{onehalfspacing}

  \datedsubsection{\textbf{飞书审计}, 字节跳动实习}{2021年}
  \role{飞书商业化, 服务端开发}{}
  \begin{onehalfspacing}
    面向企业内部审计需求构建的飞书集成解决方案。针对当前飞书Admin管理与审计能力耦合的现状,基于开放平台能力设计符合\textbf{三元管理原则}的审计应用。基于“飞书审批”事件订阅,以及定时任务与乐观锁实现\textbf{延迟任务}机制,完成用户\textbf{应用访问权限动态管理}、行为与消息审计任务的异步处理;独自完成数据库设计,实现用户、审批、审计模块功能接口。
  \end{onehalfspacing}

  % \datedsubsection{\textbf{智药科技官网}, 智能药物线上筛选平台}{2019年}
  % \role{服务端开发}{}
  % \begin{onehalfspacing}
  %   具有药物筛选功能的\textbf{OMS应用}。独自完成数据库设计,实现用户、订单、文件三大模块功能接口,以及后台管理界面。
  %   基于\textbf{RBAC}实现用户权限管理;通过\textbf{JWT}实现前后端分离;部署多台任务处理集群,通过\textbf{消息队列}完成异步任务处理与消息推送。
  % \end{onehalfspacing}

% \section{\faInfo\ 相关技能}
%   \begin{onehalfspacing}
%     \begin{itemize}
%       \item 熟悉Java编程语言与JVM虚拟机,熟悉服务端编程,能编写规范低耦合的代码
%       \item 熟悉MySQL数据库,掌握基本的SQL优化、索引优化,了解数据库的事务机制、隔离级别
%       \item 了解常用Linux命令,掌握常用的运维知识,能独立维护服务器
%       \item 了解Raft分布式一致性算法基本原理
%       \item 了解Git命令,能解决开发中遇到的常见问题
%     \end{itemize}
%   \end{onehalfspacing}

\end{document}